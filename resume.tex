%%%%%%%%%%%%%%%%%%%%%%%%%%%%%%%%%%%%%%%%%%%%%%%%%%%%%%%%%%%%
% Vedic Nerd - Resume LaTeX Template
%
% Original author:
% Ganesh Mohan (http://about.me/ganesh.mohan)
%
% Github repository:
% https://github.com/ganesh2shiv/resume-latex-template
% 
%%%%%%%%%%%%%%%%%%%%%%%%%%%%%%%%%%%%%%%%%%%%%%%%%%%%%%%%%%%%

\documentclass{article}

\usepackage{fullpage}
%\usepackage{amsmath}
\usepackage{amssymb}
\usepackage[T1]{fontenc}
\usepackage{fancyhdr}
\usepackage{lastpage}
\usepackage{hyperref}
% DEFINITIONS FOR RESUME

\textheight=10in
\pagestyle{fancy}
\raggedright
\fancyhf{}
\renewcommand{\headrulewidth}{0pt}

\setlength{\hoffset}{-2pt}
\setlength{\footskip}{20pt}

\def\bull{\vrule height 0.8ex width .7ex depth -.1ex }

\newcommand{\contact}[3]{
\vspace*{5pt}
\begin{center}
{\Huge \scshape {#1}}\\
\vspace{3pt}
#2 
\vspace{2pt}
#3
\end{center}
\vspace*{-8pt}
}

\newcommand{\header}[1]{{
\hspace*{-15pt}\vspace*{6pt} \textsc{#1}} \vspace*{-6pt} 
\lineunder
}

\newcommand{\lineunder}{
\vspace*{-8pt} \\ \hspace*{-18pt} 
\hrulefill \\
}

\newcommand{\content}{
\vspace*{2pt}%
}

\newcommand{\college}[5]{\vspace*{2pt}% 
\textbf{#1} \labelitemi #2 \labelitemi #3 \hfill #4 \\ #5 
\vspace*{5pt}
}

\newcommand{\school}[4]{
\textbf{#1} \labelitemi #2 \hfill #3 \\ #4 \vspace*{5pt}
}

\newcommand{\employer}[4]{{
\vspace*{2pt}%
\textbf{#1} #2 \hfill #3\\ #4 \vspace*{2pt}}
}

\newcommand{\project}[4]{{
\vspace*{2pt}% 
\textbf{#1} #2 \hfill #3\\ \textit{#4} \vspace*{2pt}}
}

\renewcommand{\labelitemi}{
%$\vcenter{\hbox{\tiny\textbullet}}$\hspace*{3pt}
	\raisebox{0.3ex}{\tiny\textbullet}
}

\renewcommand{\labelitemii}{
%$\vcenter{\hbox{\tiny$\bullet$}}$\hspace*{-3pt}
	\raisebox{0.3ex}{\tiny\textbullet}
}

\newenvironment{bullet-list-major}{
\begin{list}{\labelitemii}{\setlength\leftmargin{3pt} 
\topsep 0pt \itemsep -2pt}}{\vspace*{4pt}\end{list}
}

\newenvironment{bullet-list-minor}{
\begin{list}{\labelitemii}{\setlength\leftmargin{15pt} 
\topsep 0pt \itemsep -2pt}}{\vspace*{4pt}\end{list}
}

\cfoot{
Last updated: Jan 21, 2022
}

\rfoot{
Page $\thepage\hspace*{3pt}\vert\hspace*{3pt}\pageref{LastPage}$
}

% END RESUME DEFINITIONS

\begin{document}

\small
\smallskip
\vspace*{-44pt}

\contact{Deebul Nair}
{Wacholderweg 9, Sankt Augustin, Germany, 53757\\}
{(+49) 015218672925 \labelitemi deebuls@gmail.com}
\vspace{15pt}
%\header{Personal Profile}
%    \content{A detail oriented and experienced Android native application developer with a stellar record of corporate client satisfaction. Always willing to learn \& experiment, happy to work both independently and as part of a team.\vspace{5pt}}

\vspace*{4pt}%
%\header{Objective}
%    \content{To work in an environment that provides a challenging and rewarding career, ensuring a high level job satisfaction.\vspace{5pt}}


\vspace*{4pt}%
\header{Work Experience}
    \employer{Senior Robotics Software Developer}{-- ST Engineering Applied Solutions }{2022 -- present}
    {Singapore Technologies, Frankfurt}
	\begin{bullet-list-minor}
	\item Developed and Integrated custom navigation stack for construction robots on top of ROS architecture.
	\item Developed and Integrated custom navigation stack for construction robots on top of ROS architecture.
	\item  Deployed robot control software on ARM based embedded systems.
        \item  Developed and deployed computer vision algorithms for water spray detection for construction robot. 
        \item  Demonstrated expertise in C/C++/Python programming, particularly in an embedded environment, to ensure efficient and high-performance code execution.
        \item  Temporal sensor fusion in DNN with different probabilistic methods Kalman Filter, Particle Filter. 
        \item  Ported robot software from ROS1 to ROS2.
        \item  Integrated the software with various ROS/ROS2 modules, ensuring seamless communication and compatibility with existing systems.
        \item  Wrote detailed documentation, including design specifications, user manuals, and technical guides, to facilitate seamless knowledge transfer and maintain system integrity.
        \item  Expertly leveraged computer vision frameworks and libraries such as Torch, PyTorch, Tensorflow, Tflite, ONNX, and OpenCV to develop and deploy robust applications.
        \item  Collaborated effectively with a team of engineers, fostering an environment of innovation and teamwork, while also being capable of independently driving projects forward.
        \end{bullet-list-minor}

    \employer{Senior Research Assistant}{-- Sesame Project}{2018 -- 2022}
    {Bonn-Rhein-Sieg University of Applied Sciences}
	\begin{bullet-list-minor}
		%\item  Led the technical design phase, collaborating with a cross-functional team to determine the optimal architecture and system requirements.
		\item  Defined and architected a software system for implementing deep learning and computer vision algorithms on embedded systems(Yolov8 / FOMO/ Resnet8).
		\item  Implemented the system, including architecture modifications and quantization, to ensure efficient processing and execution on embedded platforms(Coral, Movidius, Raspberry Pi, OAK-D).
		\item  Trained and deployed deterministic uncertainty estimation for Deep Neural Networks(DNN) to improve robustness and dependability attributes.
		%\item  Developed custom data collection rigs and smart annotation software of data.
		\item  Integrated the software with various ROS/ROS2 modules, ensuring seamless communication and compatibility with existing systems.
	%	\item  Wrote detailed documentation, including design specifications, user manuals, and technical guides, to facilitate seamless knowledge transfer and maintain system integrity.
		%\item  Utilized strong understanding of machine learning and computer vision algorithms, including DNN, to design and optimize algorithms for efficient implementation on embedded systems.
	%	\item  Expertly leveraged computer vision frameworks and libraries such as Torch, PyTorch, Tensorflow, Tflite, ONNX, and OpenCV to develop and deploy robust applications.
	%	\item  Demonstrated expertise in C/C++/Python programming, particularly in an embedded environment, to ensure efficient and high-performance code execution.
	%	\item  Temporal sensor fusion in DNN with different probabilistic methods Kalman Filter, Particle Filter. 
	%	\item  Collaborated effectively with a team of engineers, fostering an environment of innovation and teamwork, while also being capable of independently driving projects forward.
		\item  Open source contribution for probabilistic programming languages.
		\item  Experiment Design and statistic test using Bayesian analysis for experiment reporting.
    \end{bullet-list-minor}
    
    \employer{Robotics Team Leader}{-- b-it-bots Team}{2017 -- 2022}
    {Bonn-Rhein-Sieg University of Applied Sciences}
	\begin{bullet-list-minor}
		\item Managing a team of software developers and roboticist.
		\item Architecture Design, Software development(c++, python), Testing, CI, H/W Integration.
		\item Integration of deep learning algorithms with ROS/ROS2 architecture.
		\item Deep learning networks on embedded boards (movidius/OAK-D).
		\item Maintainer for university open-source projects <https://github.com/b-it-bots>.
		\item Maintenance of multiple embedded robots like Youbot, Robile, Toyota HSR, Kinova arm.
		\item Navigation stack (ros\_navigation, nav2) deployment on multiple robots.
		\item State machine development using SMACH and Behaviour Tree
		\item Custom localization algorithm development and integration on robots with 3D lidar.
    \end{bullet-list-minor}
    
%    \employer{Project Co-ordinator}{-- SciRoc Project}{2017 -- 2022}
%    {Bonn-Rhein-Sieg University of Applied Sciences}
%	\begin{bullet-list-minor}
%	\item Robotics benchmarking protocol definition for smart city scenario \url{https://sciroc.org/}. 
%    \item Conducting robotics benchmarking events.
%    \item Data collection and reporting of robot tasks.
%    \end{bullet-list-minor}
    

    \employer{Research Assistant}{-- DigiKlausur}{2016 -- 2017}
    {Prof. Paul Pl\"{o}ger, Bonn-Rhein-Sieg University of Applied Sciences}
    \begin{bullet-list-minor}
\item Docker, Kubernetes cluster deployment for the Jupyter notebook based electronic examination.
\item Deployed jupyterhub server on google cloud and OpenStack.
    %\end{bullet-list-minor}

    %\employer{Teaching Assistant}{-- Mathematics for Robotics and Control}{2016}
    %{Prof. Paul Pl\"{o}ger, Bonn-Rhein-Sieg University of Applied Sciences}
    %\begin{bullet-list-minor}
    %\item  Prepared  and  presented  lectures  and  recitations,  supported  term  projects,  helped  students  with  course  materials, and %graded homework
    %\item Developed Jupyter notebook based assignments.
    %\end{bullet-list-minor}

    %\employer{Research Software Developer}{-- Machine learning tool development}{2015 -- 2016}
    %{Prof. Gerhard Kraetzschmar, Bonn-Rhein-Sieg University of Applied Sciences}
    %\begin{bullet-list-minor}
    \item Explainable AI methods for the \emph{Decision Tree} machine learning algorithm.
    \item Developed tool to visualize decision tree output developed using scikit-sklearn python package.
    \item \url{https://github.com/deebuls/decision\_tree\_visualize}
    \end{bullet-list-minor}

    \employer{Senior Embedded Software Engineer}{-- IMO Project}{2010 -- 2014}
    {Rockwell Collins, Inc}
    \begin{bullet-list-minor}
    \item Software development of the smart router for AIRBUS A350.
    \item Responsible for implementation of communication manager which co-ordinated availability of 
	    different communication medium(LAN, WLAN, GSM, SATCOM) and provided the user with uninterrupted service.
    \item Involved in complete life cycle of software development from
requirement writing(IBM Rational DOORS), designing(UML), development(C++), testing and software integration.
    %\end{bullet-list-minor}

    %\employer{Senior Embedded Software Engineer}{-- ARINC  822}{2012 -- 2013}
    %{Rockwell Collins, Inc}
    %\begin{bullet-list-minor}
    \item Implemented the proposed ARINC specification 822 - Air/ground
Wireless Communication (Gatelink).
    \item Software development in C++ for communicating with the WLAN controllers and
access point.
   % \item SDLC using IBM Rational DOORS
    %\end{bullet-list-minor}


    %\employer{Embedded Software Engineer}{Device Driver }{2011 -- 2012}
    %{Rockwell Collins, Inc}
    %\begin{bullet-list-minor}
    \item Developed and maintained Linux device driver for custom FPGA chip on PowerPC MPC8572.
    \item Developed and integrated Linux device driver for I2C protocol chip PCA555.
    %\end{bullet-list-minor}

    %\employer{Associate Engineer}{-- Board Support Package}{2010 -- 2011}
    %{Rockwell Collins, Inc}
    %\begin{bullet-list-minor}
    \item Developed board support package for a Windriver Linux based customized x86 board.
   % \item Operating system selection and configuration.
    \item Boot loader selection and customizing to the board.
   % \item Compiling device drivers for the peripherals.
    \end{bullet-list-minor}

%%%%% NEW PAGE
\newpage
%%%%% NEW PAGE


\vspace*{4pt}%
\header{Technical Skills}
    \begin{bullet-list-major}
    \item Programming Languages: C, C++, Python
    \vspace{2pt}
    \item Platforms \& Frameworks: Robot Operating Systems(ROS), Tensorflow, PyTorch, Scikit
    \vspace{2pt}
    \item Tools \& Libraries: Git, Vim, Github actions, Docker, Travis, Lxc container
    \vspace{2pt}
    \item Operating Systems: Ubuntu, Debian, Linux, VxWorks, Windriver Linux
    \vspace{2pt}
\item Standards:  DO-178B, ISO 26262, FMEA, AIRINC 429
    \vspace{2pt}
\item Software Development: Scrum, Jira, Git, Github
    \vspace{2pt}
\item Computer Vision Libraries: OpenCV, PCL, Kornia
    \vspace{2pt}
\item Build Systems: Autotools, Make, CMake
    \end{bullet-list-major}


\vspace*{4pt}%
\header{Open Source Contributions}

\employer{b-it-bots}{-- \url{https://github.com/b-it-bots/mas\_industrial\_robotics}}{}{}
\begin{bullet-list-minor}
    \item ROS based software architecture for youBot industrial mobile robot.
    \item contributor and administrator
    \item code base for youBot robot used in international competitions
\end{bullet-list-minor}
\employer{sa\_tool\_python}{-- \url{https://github.com/deebuls/sa_tool_python} }{}
    {}
    \begin{bullet-list-minor}
    \item python based tool for fault diagnosis in robots.
    \item contributor and maintainer
    \item used as a teaching tool in the course ``Fault detection and diagnosis''
    \end{bullet-list-minor}

\employer{BayesPy}{-- \url{https://github.com/bayespy/bayespy/}}{}{}    
    \begin{bullet-list-minor}
    \item tools for Bayesian inference with Python
    \item contributor
    \end{bullet-list-minor}
    
  
\vspace*{4pt}%
\header{Publications}
\begin{bullet-list-major}
 \item \textbf{D. Nair}, N. Hochgeschwender, M. Olivares-Mendez,  ``\textbf{Maximum Likelihood Uncertainty Estimation: Robustness to Outliers}'', in proceedings of the Workshop on Artificial Intelligence Safety 2022 (SafeAI 2022)
co-located with the Thirty-Sixth AAAI Conference on Artificial Intelligence (AAAI 2022) 
 \item A. Padalkar, M. Wasil, S. Mahajan, R. Kumar, D. Bakaraniya, R. Shirodkar, H. Andradi, D. Padmanabhan, C. Wiesse, A. Abdelrahman, S. Chavan, N. Gurulingan, \textbf{D. Nair}, S. Thoduka, I. Awaad, S. Schneider, P. G. Pl\"{o}ger, and G. K. Kraetzschmar, ``\textbf{b-it-bots: Our Approach for Autonomous Robotics in Industrial Environments}'', in Proceedings of the 23rd RoboCup International Symposium, Sydney, Australia, 2019. 
 \item K. Jeeveswaran, M. Muthuraja, \textbf{D. Nair}, and P. G. Pl\"{o}ger, ``\textbf{Using Active Learning for Assisted Short Answer Grading}'' in ICML Workshop on Real World Experiment Design and Active Learning, , 2020. 
\end{bullet-list-major}

\vspace*{4pt}%
\header{Education}
    \school{University of Luxembourg}{Luxembourg city, Luxembourg}{Present}
    {\textit{Doctoral Candidate} \labelitemi SpaceR Lab}
    
    \school{Bonn-Rhein-Sieg University Applied Sciences}{Sankt Augustin, Germany}{ 2017}
    {\textit{Master of Science } \labelitemi Autonomous Systems}
    
    \school{Centre for Development of Advanced Computing}{Hyderabad, India}{2010}
    {\textit{Post Graduate Diploma} \labelitemi Embedded Systems Design }

    \school{Xavier's Institute of Technology (Mumbai University)}{Mumbai, India}{2009}
    {\textit{Bachelor of Technology} \labelitemi Electronics and Communication Engineering }

\vspace*{4pt}%
\header{Achievements}
    \begin{bullet-list-major}
    \item 1st place in Robocup@work: RoboCup 2019, Sydney, Australia
    \item 1st place in Robocup@work: German Open 2019, Magdeburg, Germany
    \end{bullet-list-major}

\vspace*{4pt}%
\header{Areas of Interest}
    \content{\labelitemi Robotics design \& integration \labelitemi Deep learning \& Architectures \labelitemi Embedded system design  \vspace{5pt}}

\end{document}
